%! Author = rickr
%! Date = 9/1/2021
\vspace{1cm}
\section*{Problem 3}
	Decide True or False for each of the followings. You MUST briefly justify your answer.\\
	
	\noindent
	\textbf{\underline{Satisfiability}:}\\
	\underline{Instance}: Set U of variables, collection C of clauses over U.\\
	\underline{Question}: Is there a satisfying truth assignment for C?\\
	
	\noindent
	(a) If P $\neq$ NP, then no problem in NP can be solved in polynomial time deterministically.\\
	(b) If a decision problem A is NP-complete, proving that A is reducible to B, in polynomial time, is sufficient to show that B is NP-complete.\\
	(c) It is known that SAT (Satisfiability) is NP-complete, and 3SAT (all clauses have size 3) is NP-complete. 1SAT (all clauses have size 1) is also NP-complete.
	
\section*{Solution:}
	\subsection*{(a)}
		\textbf{False}\\
		The class of problems defined as P consists of the set of decision problems $\Pi$ that can be solved by a deterministic Turing machine using a polynomial amount of computation time. 
		This class of problems is known to be a subset of NP, that is $P\subseteq NP$, and therefore, every decision problem solvable by a polynomial time deterministic algorithm is also solvable by a  polynomial time nondeterministic algorithm. 
		Suppose that the decision problem $\Pi \in P$ has a polynomial deterministic algorithm $A$. 
		We can obtain a polynomial time nondeterministic algorithm for $\Pi$ by using $A$ as the checking stage and ignoring the guess. Therefore, $\Pi \in P \implies \Pi \in NP$.
	
	\newpage
	\subsection*{(b)}
		\textbf{False}\\
		The process of devising an NP-completeness proof for a decision problem $\Pi$ consists of 4 stages:\\
		
		\noindent
		i) Show that $\Pi \in NP$\\
		ii) Select a known NP-complete problem $\Pi'$\\
		iii) Construct a transformation $f: \Pi' \rightarrow \Pi$\\
		iv) Prove that $f$ is a polynomial time transformation\\
		
		\noindent
		For the given scenario, conditions ii, iii, and iv are satisfied. 
		However it was never stated that $B \in NP$. Therefore we cannot yet say that $B$ is NP-complete. 
		
	\subsection*{(c)}
		\textbf{False}\\
		Let $U = \{u_1, u_2 \dots u_n\}$ be a set of boolean variables and let the function $t$ be the truth assignment of $U$ such that $t:U\rightarrow\{T,F\}$. 
		If we allow the set of literals over $U$ to form a clause over $U$ in the form $C = \{\{u_1\}, \{u_2\} \dots \{u_n\}\}$, then a satisfying truth assignment will take the form $t(u_1) = t(u_2)= \dots = t(u_n) = T$. 
		It a simple matter of scanning $C$ for an instance of $\neg u_i$ from $1\leq i \leq n$. 
		Therefore, since 1SAT is solvable in a time on the order of $O(n)$ we can say that 1SAT $\in P$.