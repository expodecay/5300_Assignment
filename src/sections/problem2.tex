%! Author = rickr
%! Date = 9/1/2021

\section*{Problem 2}
    \textbf{\underline{1-to-2 PARTITION}}:\\
    \underline{Instance}: A finite set of positive integers Z = { z1 , z2 , ... , zn }.\\
    \underline{Question}: Is there a subset Z' of Z such that
    Sum of all numbers in Z' = 2 x Sum of all numbers in Z-Z'\\
    
    \noindent
    (a) Obtain the dynamic programming functional equation to solve the 1-to-2 PARTITION problem.\\
    (b) Give an algorithm to implement your functional equation.\\
    (c) Give an example of 5 numbers with a total of 21 as an input instance for 1-to-2 PARTITION
    
     problem, and show how your algorithm works on this input instance.\\
    (d) What is the complexity of your algorithm?
\section*{Solution}
	\subsection*{(a)}
		Consider the case of $\sum Z = \sum Z-Z'$. 
		For a solution to exist, two conditions must be satisfied:\\
		$\bullet$ The sum of the total number integers $\sum_{i=1}^{n} z_i$ must be even. 
		If it is odd, then the set cannot be divided into two equal subsets. \\
		$\bullet$ If $\exists$ a sum $S_1 = \sum_{i=1}^{m} z_i = \frac{1}{2}\sum_{i=1}^{n} z_i $, where $m < n$, then we are guaranteed to have the remaining $Z-Z'$ elements equal to $S_2 = \frac{1}{2}\sum_{i=1}^{n} z_i$. 
		The set can be broken into two equal sums.\\
		
		\noindent
		Consider the example of $Z = \{1,3,5,11\}$. 
		The fist condition passes with $\sum_{i=1}^{n} z_i = 20$ \ding{51} however, the second condition requires us to find a subset whose sum is equal to $S_1 = \frac{1}{2}\sum_{i=1}^{n} z_i = 10$. 
		The best we can do in this case is $Z_1 = \{1,3,5\}\rightarrow S_1 = 9\neq 10$ and $Z_2 = \{11\}\rightarrow S_2 = 11$ \ding{55}. 
		$\therefore$ this $Z$ input will return false.\\

		\noindent
		Now consider $Z = \{3,1,1,2,2,1\}$. 
		The first condition passes with $\sum_{i=1}^{n} z_i = 10$ \ding{51}, and the second condition passes since $Z_1 = \{3,1,1\} \rightarrow S_1 = 5$ and $Z_2 = \{2,2,1\} \rightarrow S_2 = 5$ \ding{51}. 
		$\therefore$ this $Z$ input will return true.

		\begin{center}
			\scalebox{0.75}{%}
				\begin{tabular}{|c|c|c|c|c|c|c|}
					\hline
					\multicolumn{7}{|c|}{Initial}\\
					\hline
					\backslashbox{Z $\downarrow$ }{Sum (S) $\rightarrow$}&0&1 &2 &3 &4 & 5 \\
					\hline
						  &1 &0 &0 &0 &0 &0\\
					\hline
					$z_1$=3&1 & & & & &\\
					\hline
					$z_2$=1&1 & & & & &\\
					\hline
					$z_3$=1&1 & & & & &\\
					\hline
					$z_4$=2&1 & & & & &\\
					\hline
					$z_5$=2&1 & & & & &\\
					\hline
					$z_6$=1&1 & & & & &\\
					\hline
			\end{tabular}}\quad$\Rightarrow$\quad
			\scalebox{0.75}{%}
				\begin{tabular}{|c|c|c|c|c|c|c|}
					\hline
					\multicolumn{7}{|c|}{Final}\\
					\hline
					\backslashbox{Z $\downarrow$ }{Sum (S) $\rightarrow$}&0&1 &2 &3 &4 & 5 \\
					\hline
					&1 &0 &0 &0 &0 &0\\
					\hline
					$z_1$=3&1 &0&0&1&0&0\\
					\hline
					$z_2$=1&1 &1&0&1&1&0\\
					\hline
					$z_3$=1&1 &1&1&1&1&1\\
					\hline
					$z_4$=2&1 &1&1&1&1&1\\
					\hline
					$z_5$=2&1 &1&1&1&1&1\\
					\hline
					$z_6$=1&1 &1&1&1&1&\textcolor{red}{1}\\
					\hline
			\end{tabular}}
		\end{center}
		Starting with the first empty cell of the matrix:\\
		$1-3=\textit{undefined}\rightarrow\text{false}\therefore$ we cannot form the sum=1 using integer 3\\
		$2-3=\textit{undefined}\rightarrow\text{false}\therefore$ we cannot form the sum=2 using integer 3\\
		$3-3=0\rightarrow\text{true\footnote[1]{}}\therefore$ we CAN form S=3 using z=3 (populate cell with 1) Note: s(3) = 0 and s(0) = 1\\\footnotetext[1]{We take a step up to the previous row and traverse to the left until we reach the column value that is equal to the difference between the sum and integer.}
		\hspace{-0.25cm}
		$4-3=1\rightarrow\text{false}$ Note: s(4) = 0 and s(1) = 0\\
		$5-3=2\rightarrow\text{false}$ Note: s(5) = 0 and s(2) = 0\\
		The emerging pattern can be summarized as: 
		if the ith-1 cell is equal to true OR the difference between s and z for that particular row and column is true, then we take true. ie if $f(s) = 1 \quad || \quad f(s-z) = 1$. Therefore, our functional equation can be expressed as
		\begin{equation*}
			\boxed{f_i[s,z_i] = max\{f_{i-1}(s), f_{i-1}(s-z_i)\}}
		\end{equation*}
		If we find the last cell to be true, then it IS possible to form the value $\frac{S}{2}$ with the given assortment of Z. 
		This relation also holds for the case of $\sum Z' = \sum 2(Z-Z')$ since we are just moving the partition while preserving the total sum. 
		However we do need to change the first condition to only allow a target sum divisible by 3, in which case the second condition is changed to $S_1 = \frac{2}{3}\sum_{i=1}^{n} z_i$ and $S_2 = \frac{1}{3}\sum_{i=1}^{n} z_i$. 
	
		\subsection*{(b)}
		\vspace{-0.5cm}
			\begin{algorithm}[H]
				\caption{1-to-2 PARTITION}
				\begin{algorithmic}[1]
					\Procedure{{{partition}}}{\textbf{int} Z[n]}
					
						\State int sum = 0, s = 0
						
						\For{i $\leftarrow$ 1 to n} \Comment{$\mathcal{O}(n)$}
							\State sum += Z[i]
						\EndFor
						
						\If{sum \% 3 $\neq$ 0}
							\State\textbf{return} false \Comment{Check that the first condition is satisfied}
						\EndIf
						
						\State s = sum / 3
						\State int f[n][s] \Comment{Initialize a matrix}
						
						\For{j $\leftarrow$ 1 to s} 
							\State f[0][j] = 0
							\State f[j][0] = 1
						\EndFor
						
						\For{i $\leftarrow$ 1 to n}
							\For{j $\leftarrow$ 0 to s}
								\If{j $<$ Z[i-1]}
									\State f[i][j] = f[i-1][j] \Comment{The cell above}
									\Else
										\State f[i][j] = \textit{max}\{f[i-1][j] , f[i-1][j-Z[i-1]]\}
								\EndIf
							\EndFor
						\EndFor
						\State \textbf{return} f[n][s]
					\EndProcedure
				\end{algorithmic}
			\end{algorithm}

		\subsection*{(c)}
			Let $Z = 1,2,3,5,10$, $sum = \sum_{i = 1}^{n} z[i] = 21$, and $M = \frac{1}{3}sum = 7$
				\begin{center}
				\scalebox{0.75}{%}
					\begin{tabular}{|c|c|c|c|c|c|c|c|c|}
						\hline
						\backslashbox{Z $\downarrow$ }{Sum (S) $\rightarrow$}&0&1 &2 &3 &4 & 5 & 6 & 7\\
						\hline
						&1 &0 &0 &0 &0 &0& 0 & 0 \\
						\hline
						$z_1$=1&1 &1 &0 &0 &0 &0 &0 &0\\
						\hline
						$z_2$=2&1 &1 &1 &1 &0 &0 &0 &0\\
						\hline
						$z_3$=3&1 &1 &1 &1 &1 &1 &1 &0\\
						\hline
						$z_4$=5&1 &1 &1 &1 &1 &1 &1 &1\\
						\hline
						$z_5$=10&1 &1 &1 &1 &1 &1 &1 &1\\
						\hline
				\end{tabular}}
			\end{center}
			Since the algorithm halts with a 1 on cell $f(z_5, s_7)$, the instance will return \textit{true}. 
			The algorithm can be slightly improved by returning a \textit{true} at the first occurrence of the target sum $s[7] \leftarrow 1$ because we know at that point that the remaining elements will sum to $\frac{2}{3}$ total sum. 
		\subsection*{(d)}
			Since the algorithm must iterate through $\frac{1}{3}sum$ columns and $Z$  rows, the time complexity must be on the order of $O(Tn)$, where $T = \frac{sum}{3}$.
		
		
		
		