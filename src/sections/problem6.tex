%! Author = rickr
%! Date = 9/1/2021

\section*{Problem 6}
	\textbf{\underline{Optimization PS(3) Problem:}} Given a set of n program and three storage devices. 
	Let $s_i$ be the amount of storage needed to store the $i^{th}$ program. 
	Let L be the storage capacity of each disk. Determine the maximum number of these n programs that can be stores on the three disks (without splitting a program over the disks).\\
	
	\noindent
	Use the Approximation PS Algorithm given in the class for the PS(3) problem given above. 
	Show that the following is true.\\
	
	\noindent
	Let the approximation PS algorithm returns a number $C$, and let $C^*$ be the optimal (maximum) number of programs that can be stores on the three disks.\\
	(a) Show that the above approximation PS algorithm gives the performance ratio of\[C^* \leq (C+2) \quad\text{ OR }\quad \frac{C^*}{C} \leq 1 + \frac{2}{C}\]\\
	
	\noindent
	(b) Give an example that achieves the performance ratio of $C^* = (C+2)$.
\section*{Solution:}
	\begin{algorithm}[H]
		\caption{PStore($s, n$, L)}
		\begin{algorithmic}[1]
			\State // Assume that s[i] $\leq$ s[i+1] \Comment{non-decreasing order}
			\State i = 0; c = 0 \Comment{c: count the number of stored programs}
			\For{j = 1 to 2}
				\State sum = 0
				\While{sum + s[i] $\leq$ L}
					\State store $i^{th}$ program into $j^{th}$ device
					\State sum += s[i]
					\State i++; c++
					\If{i $>$n}
						\State \Return
					\EndIf
				\EndWhile
			\EndFor
		\end{algorithmic}
	\end{algorithm}

	\subsection*{(a)}
		Let $C^*$ be the maximum number of programs that can be stored on \textit{\underline{two} disks each of length $L$} and let $C$ be the number of programs stored using PStore.
		Assume that $k$ programs are stored when PStore is used so that $C=k$. 
		If we consider the case of storing programs in non-decreasing order on a \textit{\underline{single} disk of length $2L$}, then the number of programs stored will be $\alpha$. 
		We can say that 
		\[\alpha \geq C^* \text{\quad and \quad} \sum_{1}^{\alpha}s[i] \leq 2L\]  
		If we let $j$ be the largest index such that
		\[\sum_{1}^{j}s[i] \leq L \text{\quad and \quad } \sum_{1}^{j+1}s[i] > L \] 
		then we can say $j\leq \alpha$ and that PStore assigns the first $j$ programs to disk 1. 
		We can also say that 
		\begin{equation*}
			\sum_{i = j+1}^{\alpha - 1}s[i] \leq \sum_{i = j+2}^{\alpha}s[i] \leq L
		\end{equation*}
		Hence, PStore will assign at least the first $j+1, j+2, \dots \alpha -1$ programs to disk 2. 
		This relation holds for any number of programs stored on any number of disks. 
		If we consider an arbitrary number of disks $k$ and load them with programs in a non-decreasing order, then eventually the programs still be stored on the last $k^{th}-1$ and $k^{th}$ disk. 
		We will find that the first batch of $j'$ programs will be stored on the $k^{th}-1$ disk, and that \textit{at least} the last $j'+1, j'+2, \dots \alpha' -1$ will be stored on the $k^{th}$ disk. 
		So for each \textit{pair} of disks, space for one program is lost. 
		Therefore, 
		\begin{equation*}
			C^* \leq C + k - 1
		\end{equation*}
		For the case of $k = 3$ disks, space for one program is lost between disk 1 and 2, and another space is lost between disk 2 and 3, and the approximation algorithm PStore obtains a performance ratio of 
		\begin{equation*}
			\frac{C^*}{C} \leq 1 + \frac{2}{C}
		\end{equation*} 
	\subsection*{(b)}
		Let $L=8$, $n=9$, and $s[1], s[2], \dots, s[n] = \{2, 2, 2, 3, 3, 3, 3, 4, 4\}$\\
		If we use one disk of size 24L, then all programs will fit, however using 3 disks of size 8 each, there will be no room for s[8] and s[9] under the PStore routine. 
		
		\usetikzlibrary{positioning}
		\tikzset{
			gray box/.style={
				fill=gray!15,
				draw=gray,
				minimum width={2*#1ex},
				minimum height={2em},
			},
			annotation/.style={
				anchor=south,
			}
		}
		\begin{center}
			\begin{tikzpicture}[node distance=-0.5pt]
				\node [] (p0) {disk 1 \qquad};
				\node [gray box=8, right=of p0] (p1) {s[1]};
				\node [gray box=8, right=of p1] (p2) {s[2]};
				\node [gray box=8, right=of p2] (p3) {s[3]};
				\node [gray box=8, right=of p3] (p4) {};
				
				\node [annotation] at (p1.north west) {0};
				\node [annotation] at (p1.north east) {2};
				\node [annotation] at (p2.north east) {4};
				\node [annotation] at (p3.north east) {6};
				\node [annotation] at (p4.north east) {8};
			\end{tikzpicture}
		\end{center}
		\begin{center}
			\begin{tikzpicture}[node distance=-0.5pt]
				\node [] (p0) {disk 2 \qquad};
				\node [gray box=12, right=of p0] (p1) {s[4]};
				\node [gray box=12, right=of p1] (p2) {s[5]};
				\node [gray box=8, right=of p2] (p3) {};
				
				\node [annotation] at (p1.north west) {0};
				\node [annotation] at (p1.north east) {3};
				\node [annotation] at (p2.north east) {6};
				\node [annotation] at (p3.north east) {8};
			\end{tikzpicture}
		\end{center}
		\begin{center}
			\begin{tikzpicture}[node distance=-0.5pt]
				\node [] (p0) {disk 3 \qquad};
				\node [gray box=12, right=of p0] (p1) {s[6]};
				\node [gray box=12, right=of p1] (p2) {s[7]};
				\node [gray box=8, right=of p2] (p3) {};

				\node [annotation] at (p1.north west) {0};
				\node [annotation] at (p1.north east) {3};
				\node [annotation] at (p2.north east) {6};
				\node [annotation] at (p3.north east) {8};
			\end{tikzpicture}
		\end{center}
		